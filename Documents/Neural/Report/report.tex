\documentclass[10pt]{article}
\author{
	Brendan Case{\footnote{1801421}} 
	\and Guy Coop{\footnote{gtc434 - 1447634}}
	\and Vasileios Mizaridis{\footnote{1844216}}
	\and Priyanka Mohata{\footnote{1341274}}
	\and Liangye Yu{\footnote{1810736}}
	}
	
\title{Regional Image Recognition using R-CNN}
\date{\today}
\usepackage{cite}

\begin{document}
 \maketitle
 
\section*{Abstract}
Image Recognition has, in recent history been relatively "solved" in the sense that algorithms have now been recorded outperforming humans in simple image recogntion tasks. However regional image recognition where the algorithm is tasked with recognizing multiple images inside a whole chaotic scene is still an emerging field. Our team analysed the most promising methods of regional image recognition, and implemented our own solution to a simplified regional recognition task. Our solution {\it was able to locate and classify objects inside an image with a 92\% accuracy of locations and a 100\% accuracy of classification}.


\newpage
\tableofcontents
\newpage

\section{Introduction}
For this project, our team was assigned the task of implementing a regional image recognition system.
\subsection{Data sets}
The data set provided was given in the following format:
\begin{itemize}
	\item Data: 400x400 RGB image files in .jpg format
	\item Labels: each image has a corresponding text file that describes the location of each of the predefined objects in the image. This location was given as pairs of integers describing horizontal runs of pixels that form a rectangular bounding box around the object. If the object was not present in the image it was given as [object 1 0] meaning that it had a run length of 0 pixels.
\end{itemize}
\paragraph{Training Set}
The training set contained n (image, label) pairs 

\paragraph{Test Set}
The test set contained m (image, label) pairs

\section{Design}
--Options available to us:
	- RCNN, Fast, Faster
	- Yolo
	- Selective Search
	- general backprop
	
\subsection{Regional Convolutional Neural Networks (RCNNs)}
	Given the regional nature of this task, The first option that should be analysed is "Regional Convultional Neural Networks" and their successors. There are three implementations of this algorithm that will be examined:
	\begin{itemize}
		\item R-CNN
		\item Fast R-CNN \cite{fast_rcnn}
		\item Faster R-CNN \cite{faster_rcnn}
	\end{itemize}
	
\paragraph{R-CNN}
R-CNN works by.....

\paragraph{Fast R-CNN}
Fast R-CNN \cite{fast_rcnn} is an update on the original R-CNN technique that was developed in 2015, it acheives approximately a 10x speed-up on the original. It does this by....

\paragraph{Faster R-CNN}
Faster R-CNN \cite{faster_rcnn} is another significat update on the Fast R-CNN technique that acheives another dramatic speedup. This algorithm was designed as part of an attempt at real time regional image recognition, and as such is able to operate in almost real time.
	

\subsection{"You Only Look Once" (YOLO)}
"You Only Look Once" (YOLO) was named as such because the theory centers around only performing a single pass across the image, rather than having to analyse the same data multiple times.

\subsection{Selective Search}
Selective Search is an algorithm used for regional image searching....
	

\section{Implementation}
--Using RCNN
--TensorFlow Layers

\section{Experiments}

\section{Conclusion}

\section{Description of Collaboration}
Everyone did good work.


\bibliography{report_bib}
\bibliographystyle{alpha}
\end{document}